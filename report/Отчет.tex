\documentclass[12pt,a4paper]{report}
\usepackage[utf8]{inputenc}
\usepackage[russian]{babel}
\usepackage[OT1]{fontenc}
\usepackage{amsmath}
\usepackage{amsfonts}
\usepackage{amssymb}
\usepackage{graphicx}
\usepackage{cmap}					% поиск в PDF
\usepackage{mathtext} 				% русские буквы в формулах
%\usepackage{tikz-uml}               % uml диаграммы



% Генератор текста
\usepackage{blindtext}

%------------------------------------------------------------------------------

% Подсветка синтаксиса
\usepackage{color}
\usepackage{xcolor}
\usepackage{listings}
 
 % Цвета для кода
\definecolor{string}{HTML}{B40000} % цвет строк в коде
\definecolor{comment}{HTML}{008000} % цвет комментариев в коде
\definecolor{keyword}{HTML}{1A00FF} % цвет ключевых слов в коде
\definecolor{morecomment}{HTML}{8000FF} % цвет include и других элементов в коде
\definecolor{captiontext}{HTML}{FFFFFF} % цвет текста заголовка в коде
\definecolor{captionbk}{HTML}{999999} % цвет фона заголовка в коде
\definecolor{bk}{HTML}{FFFFFF} % цвет фона в коде
\definecolor{frame}{HTML}{999999} % цвет рамки в коде
\definecolor{brackets}{HTML}{B40000} % цвет скобок в коде
 
 % Настройки отображения кода
\lstset{
language=C, % Язык кода по умолчанию
morekeywords={*,...}, % если хотите добавить ключевые слова, то добавляйте
 % Цвета
keywordstyle=\color{keyword}\ttfamily\bfseries,
stringstyle=\color{string}\ttfamily,
commentstyle=\color{comment}\ttfamily\itshape,
morecomment=[l][\color{morecomment}]{\#}, 
 % Настройки отображения     
breaklines=true, % Перенос длинных строк
basicstyle=\ttfamily\footnotesize, % Шрифт для отображения кода
backgroundcolor=\color{bk}, % Цвет фона кода
%frame=lrb,xleftmargin=\fboxsep,xrightmargin=-\fboxsep, % Рамка, подогнанная к заголовку
frame=tblr
rulecolor=\color{frame}, % Цвет рамки
tabsize=3, % Размер табуляции в пробелах
showstringspaces=false,
 % Настройка отображения номеров строк. Если не нужно, то удалите весь блок
numbers=left, % Слева отображаются номера строк
stepnumber=1, % Каждую строку нумеровать
numbersep=5pt, % Отступ от кода 
numberstyle=\small\color{black}, % Стиль написания номеров строк
 % Для отображения русского языка
extendedchars=true,
literate={Ö}{{\"O}}1
  {Ä}{{\"A}}1
  {Ü}{{\"U}}1
  {ß}{{\ss}}1
  {ü}{{\"u}}1
  {ä}{{\"a}}1
  {ö}{{\"o}}1
  {~}{{\textasciitilde}}1
  {а}{{\selectfont\char224}}1
  {б}{{\selectfont\char225}}1
  {в}{{\selectfont\char226}}1
  {г}{{\selectfont\char227}}1
  {д}{{\selectfont\char228}}1
  {е}{{\selectfont\char229}}1
  {ё}{{\"e}}1
  {ж}{{\selectfont\char230}}1
  {з}{{\selectfont\char231}}1
  {и}{{\selectfont\char232}}1
  {й}{{\selectfont\char233}}1
  {к}{{\selectfont\char234}}1
  {л}{{\selectfont\char235}}1
  {м}{{\selectfont\char236}}1
  {н}{{\selectfont\char237}}1
  {о}{{\selectfont\char238}}1
  {п}{{\selectfont\char239}}1
  {р}{{\selectfont\char240}}1
  {с}{{\selectfont\char241}}1
  {т}{{\selectfont\char242}}1
  {у}{{\selectfont\char243}}1
  {ф}{{\selectfont\char244}}1
  {х}{{\selectfont\char245}}1
  {ц}{{\selectfont\char246}}1
  {ч}{{\selectfont\char247}}1
  {ш}{{\selectfont\char248}}1
  {щ}{{\selectfont\char249}}1
  {ъ}{{\selectfont\char250}}1
  {ы}{{\selectfont\char251}}1
  {ь}{{\selectfont\char252}}1
  {э}{{\selectfont\char253}}1
  {ю}{{\selectfont\char254}}1
  {я}{{\selectfont\char255}}1
  {А}{{\selectfont\char192}}1
  {Б}{{\selectfont\char193}}1
  {В}{{\selectfont\char194}}1
  {Г}{{\selectfont\char195}}1
  {Д}{{\selectfont\char196}}1
  {Е}{{\selectfont\char197}}1
  {Ё}{{\"E}}1
  {Ж}{{\selectfont\char198}}1
  {З}{{\selectfont\char199}}1
  {И}{{\selectfont\char200}}1
  {Й}{{\selectfont\char201}}1
  {К}{{\selectfont\char202}}1
  {Л}{{\selectfont\char203}}1
  {М}{{\selectfont\char204}}1
  {Н}{{\selectfont\char205}}1
  {О}{{\selectfont\char206}}1
  {П}{{\selectfont\char207}}1
  {Р}{{\selectfont\char208}}1
  {С}{{\selectfont\char209}}1
  {Т}{{\selectfont\char210}}1
  {У}{{\selectfont\char211}}1
  {Ф}{{\selectfont\char212}}1
  {Х}{{\selectfont\char213}}1
  {Ц}{{\selectfont\char214}}1
  {Ч}{{\selectfont\char215}}1
  {Ш}{{\selectfont\char216}}1
  {Щ}{{\selectfont\char217}}1
  {Ъ}{{\selectfont\char218}}1
  {Ы}{{\selectfont\char219}}1
  {Ь}{{\selectfont\char220}}1
  {Э}{{\selectfont\char221}}1
  {Ю}{{\selectfont\char222}}1
  {Я}{{\selectfont\char223}}1
  {і}{{\selectfont\char105}}1
  {ї}{{\selectfont\char168}}1
  {є}{{\selectfont\char185}}1
  {ґ}{{\selectfont\char160}}1
  {І}{{\selectfont\char73}}1
  {Ї}{{\selectfont\char136}}1
  {Є}{{\selectfont\char153}}1
  {Ґ}{{\selectfont\char128}}1
  {\{}{{{\color{brackets}\{}}}1 % Цвет скобок {
  {\}}{{{\color{brackets}\}}}}1 % Цвет скобок }
}
 
 % Для настройки заголовка кода
\usepackage{caption}
\DeclareCaptionFont{white}{\color{сaptiontext}}
\DeclareCaptionFormat{listing}{\parbox{\linewidth}{\colorbox{сaptionbk}{\parbox{\linewidth}{#1#2#3}}\vskip-4pt}}
\captionsetup[lstlisting]{format=listing,labelfont=white,textfont=white}
\renewcommand{\lstlistingname}{Код} % Переименование Listings в нужное именование структуры

%------------------------------------------------------------------------------
\begin{document}
\title{Программирование}
\author{А. В. Власова}
\maketitle
%############################################################
\chapter{Основные конструкции языка}
\section{Задание 1}
\subsection{Задание}
Пользователь задает длину отрезка в саженях, аршинах и вершках (например, 8 саженей 2 аршина 11.4 вершка). Определить длину того же отрезка в метрах (в данном случае 19). 1 сажень = 3 аршина = 48 вершков, 1 вершок = 4.445 см.
\subsection{Теоритические сведения}

В процессе написания кода программы были использованы функции ввода/вывода \verb+prinf()+, \verb+f{scanf()+  и \verb+{puts()+, содержащиеся в стандартном заголовочном файле \verb+<stdio.h>+. 


\subsection{Проектирование}

Среда разработки: QT Creator 3.5.1 (opensource)
\\
Компилятор: GCC
\\
В программе разделены взаимодействие с пользователем и бизнес-логика.\verb+trancfer_to_meters.c+  считывает из консоли введенное пользователем количество саженей, аршинов и вершков, после чего передает их функции \verb+double trancfer_to_meters()+.\verb+trancfer_to_meter.c+ переводит полученные данные в метры. Окончательный результат выводится в консоль.
\\
Для контроля исправности программы были созданы модульные тесты, расположенные в файле \verb+tst_testtest.cpp+.

\subsection{Тестовый план и результаты тестирования}

При проведении модульного тестирования программа вызывает функцию \verb+void trancfer_test()+.  Код тестируется с помощью макроса \verb+QCOMPARE()+, который сравнивает значение, полученное в функции \verb+double trancfer_to_meters()+ и ожидаемый результат.

\subsection{Выводы}

Выполнение задания позволило освоить процесс разделения программы на взаимодействие с пользователем и бизнес-логику, а также приобрести некоторые навыки отладки кода и исправления ошибок, выводимых компилятором.


\subsection*{Листинги}

\lstinputlisting[]
{../sources/HomeWork/app/trancfer_to_meters.c}

\lstinputlisting[]
{../sources/HomeWork/app/trancfer_to_meters.h}

\lstinputlisting[]
{../sources/HomeWork/lib/trancfer_to_meter.c}

\lstinputlisting[]
{../sources/HomeWork/lib/trancfer_to_meter.h}



%############################################################

\section{Задание 2}
\subsection{Задание}
На шахматной доске стоят черный король и белые ладья и слон (ладья бьет по горизонтали и вертикали, слон — по диагоналям). Проверить, есть ли угроза королю и если есть, то от кого именно. Координаты короля, ладьи и слона вводить целыми числами.
\subsection{Теоритические сведения}
В процессе написания кода программы в файле \verb+search_threat.c+ был использован оператор ветвления \verb+if+. Также были задействованы функции ввода/вывода \verb+puts()+, \verb+scanf()+ и математическая функция \verb+abs()+.

\subsection{Проектирование}
Среда разработки: QT Creator 3.5.1 (opensource)
\\
Компилятор: GCC
\\
В программе разделены взаимодействие с пользователем и бизнес-логика. \verb+shahmati_ugroza.c+ считывает из консоли введенные пользователем координаты короля, ладьи и слона. Полученные данные передаются функции \verb+int search_threat()+, где происходит поиск угроз королю со стороны слона и ладьи. Найденные угрозы выводятся в консоль.
\\
Для контроля исправности программы были созданы модульные тесты, расположенные в файле \verb+tst_testtest.cpp+.

 
\subsection{Тестовый план и результаты тестирования}

При проведении модульного тестирования программа вызывает функцию \verb+void search_threat_test()+. Код тестируется с помощью макроса \verb+{QCOMPARE()+, который сравнивает значение, полученное в функции \verb+int search_threat()+, и ожидаемый результат. 
\subsection{Выводы}

Выполнение задания позволило освоить процесс разделения программы на взаимодействие с пользователем и бизнес-логику, а также приобрести некоторые навыки отладки кода и исправления ошибок, выводимых компилятором.

\subsection*{Листинги}

\lstinputlisting[]
{../sources/HomeWork/app/shahmati_ugroza.c}

\lstinputlisting[]
{../sources/HomeWork/app/shahmati_ugroza.h}

\lstinputlisting[]
{../sources/HomeWork/lib/search_threat.c}

\lstinputlisting[]
{../sources/HomeWork/lib/search_threat.h}


%############################################################
\chapter{Циклы}
\section{Задание 3}
\subsection{Задание}
Используя ряд $sin x = x - \frac{x3}{3!} + \frac{x5}{5!} - \frac{x7}{7!} + …$, рассчитать значение синуса x с заданной точностью (например, $10^{-3}$). Заданную точность считать достигнутой, если очередной элемент ряда меньше заданной точности.

\subsection{Теоритические сведения}
В процессе написания кода программы был использован цикл \verb+do while+ для поочередного определения слагаемых. Цикл прекращается при достижении указанной точности. Кроме того, были применены функции ввода/вывода \\verb+puts()+, \verb+scanf()+ и \verb+printf()+.
  
\subsection{Проектирование}
Среда разработки: QT Creator 3.5.1 (opensource)
\\
Компилятор: GCC
\\
В программе разделены взаимодействие с пользователем и бизнес-логика. \verb+search_sine.c+ считывает из консоли введенные пользователем значение x и точность. Полученные данные передаются функции \verb+double sinx()+, где высчитывается значение \verb+sin(x)+.
\\
Для контроля исправности программы были созданы модульные тесты, расположенные в файле \verb+tst_testtest.cpp+.


\subsection{Тестовый план и результаты тестирования}
При проведении модульного тестирования программа вызывает функцию \verb+void search_sine_test()+. Код тестируется с помощью макроса \verb+QCOMPARE()+, который сравнивает значение, полученное в функции \verb+double sinx()+, и ожидаемый результат. 

\subsection{Выводы}
Выполнение задания позволило освоить процесс создания цикла \verb+do while+, а также выяснить пути исправления некоторых ошибок, выводимых комплятором.

\subsection*{Листинги}

\lstinputlisting[]
{../sources/HomeWork/app/search_sine.c}

\lstinputlisting[]
{../sources/HomeWork/app/search_sine.h}

\lstinputlisting[]
{../sources/HomeWork/lib/sinx.c}

\lstinputlisting[]
{../sources/HomeWork/lib/sinx.h}


%############################################################

\chapter{Массивы}
\section{Задание 4}
\subsection{Задание}
Квадрат n x n состоит из прозрачных и непрозрачных маленьких квадратов. Имеется ли хотя бы один просвет по каждому из двух измерений? Вывести координаты каждого просвета.

\subsection{Теоритические сведения}

В процессе написания кода программы для работы с файлом и для динамического выделения памяти были использованы указатели. Кроме того, были применены циклы \verb+while+ и \verb+for+, оператор ветвления \verb+if+. Также использовались функции для работы с файлом \verb+fopen()+ и \verb+fclose()+, функции выделения и освобождения памяти \verb+{malloc()+ и \verb+free()+, функция чтения данных из файла \verb+fscanf()+, функция вывода \verb+printf()+.

\subsection{Проектирование}
Среда разработки: QT Creator 3.5.1 (opensource)
\\
Компилятор: GCC
\\
В программе разделены взаимодействие с пользователем и бизнес-логика. Функция \verb+void file_clear()+ считывает из файла данные и записывает их в двумерный массив. Полученные данные передаются функции \verb+void clear()+, где происходит поиск пустых строк и столбцов массива.

\subsection{Тестовый план и результаты тестирования}
После написания программы было проведено ручное тестирование.

\subsection{Выводы}
Выполнение задания позволило освоить процесс динамического выделения памяти и научиться работать с массивами.

\subsection*{Листинги}

\lstinputlisting[]
{../sources/HomeWork/app/file_clear.c}

\lstinputlisting[]
{../sources/HomeWork/app/file_clear.h}

\lstinputlisting[]
{../sources/HomeWork/lib/search_clear.c}

\lstinputlisting[]
{../sources/HomeWork/lib/search_clear.h}

%############################################################

\chapter{Строки}
\section{Задание 5}
\subsection{Задание}
Имеется большой словарь русских слов. Найти в нем слова-палиндромы, одинаково читающиеся как слева направо, так и справа налево, например, АННА, ШАЛАШ.

\subsection{Теоритические сведения}
В процессе написания кода программы была использована функция поиска длины строки \verb+strlenght()+, функции для работы с файлами \verb+fopen()+ и \verb+fclose()+. Также были применены циклы \verb+while+ и \verb+for+, оператор ветвления \\verb+if+. 
  
\subsection{Проектирование}
Среда разработки: QT Creator 3.5.1 (opensource)
\\
Компилятор: GCC
\\
В программе разделены взаимодействие с пользователем и бизнес-логика. \verb+palindromes_read.c+ считывает из файла слова и передает их функции \verb+char search_palindromes()+, где определяется, является слово палиндромом или нет. 

\subsection{Тестовый план и результаты тестирования}
После написания программы было проведено ручное тестирование.


\subsection{Выводы}
Выполнение задания позволило освоить некоторые функции заголовочного файла \verb+<string.h>+.

\subsection*{Листинги}

\lstinputlisting[]
{../sources/HomeWork/app/palindromes_read.c}

\lstinputlisting[]
{../sources/HomeWork/app/palindromes_read.h}

\lstinputlisting[]
{../sources/HomeWork/lib/palindromes_search.c}

\lstinputlisting[]
{../sources/HomeWork/lib/palindromes_search.h}

%############################################################

\chapter{Инкапсуляция}
\section{Задание 6}
\subsection{Задание}
Реализовать класс ОЧЕРЕДЬ (целых чисел, неограниченного размера). Требуемые методы: конструктор, деструктор, копирование, встать в очередь, выйти из очереди.

\subsection{Теоритические сведения}
В процессе написания кода программы был использован поток вывода \verb+cout+ из заголовочного файла \verb+<iostream>+. Для создания исключений были применены операторы \verb+try+, \verb+throw+ и \verb+catch+.
  
\subsection{Проектирование}
Среда разработки: QT Creator 3.5.1 (opensource)
\\
Компилятор: GCC
\\
Для выполнения задания был выделен класс \verb+Queue+, в котором элементы запоминаются в массив в порядке очереди. При удалении первого элемента все остальные сдвигаются на одну ячейку массива вперед. Новые элементы записываются в конец очереди.
\\
В определении класса были выделены следующие методы:
\begin{itemize}
\item конструктор \verb+Queue()+, инициализирующий свойства класса;
\item конструктор копирования \verb+Queue(const Queue &otherQueue)+, служащий для инициализации объекта класса посредствомдругого;
\item метод \verb+int plus_element(int element)+, добавляющий элементы в очередь;
\item метод \verb+int minus_element()+, удаляющий элементы из массива;
\item метод \verb+int print_queue()+, выводящий полученный массив в консоль.
\end{itemize}

\subsection{Тестовый план и результаты тестирования}
После написания программы было проведено ручное тестирование.


\subsection{Выводы}
Выполнение задания позволило понять принцип работы с классами, научиться объявлять и использовать свойства и методы класса, освоить процесс добавления ограничений.



%############################################################
\chapter{Приложение}
\subsection*{Листинги}

\lstinputlisting[]
{../sources/HomeWork/app/main.c}

\lstinputlisting[]
{../sources/HomeWork/test/tst_testtest.cpp}



\end{document}